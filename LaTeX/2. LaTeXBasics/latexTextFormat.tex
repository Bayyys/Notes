% 字体字号设置
\documentclass[10pt]{article}	% 10磅是默认正常的字体大小,即下文中的normalsize (一般只有 10、11、12 磅)
\usepackage{ctex}
\newcommand{\myfont}{\textbf{\textsf{Fancy Text}}}	% 定义命令,可在文档内使用

%正文区
\begin{document}
	%字体族设置(罗马字体、无衬线字体、打字机字体)
	
	% \textrm等是字体命令,大括号里是作用到的文字
	\textrm{Roman Family} \textsf{Scan Serif Family}\texttt{Typewriter Family}
	% \rmfamily是字体声明,后面紧跟的文字是作用到的文字
	{\rmfamily Roman Family} {\sffamily Scan Serif Family}{\ttfamily Typewriter Family}
	
	% {} 表示字体使用的作用域,否则遇到另外的字体声明才会改变
	{\sffamily Who are you?you find self on everyone around.take you as the same as others!}
	{\ttfamily Are you aiser than others?}
	
	% 字体系列设置(粗细、宽度),\textbf可以对字体加粗
	\textmd{Medium Series} \textbf{Boldface Series}	% \textmd等是字体命令,大括号里是作用到的文字
	{\mdseries Medium Series} {\bfseries Boldface Series}	% 字体声明
	
	
	% 字体形状(直立、斜体、伪斜体、小型大写)
	\textup{Upright Shape} \textit{Italic Shape} % 字体命令
	\textsl{Slanted Shape} \textsc{Small Caps Shape}
	
	{\upshape Upright Shape} {\itshape Italic Shape }	% 字体声明
	{\slshape Slanted Shape}
	{\scshape Small Caps Shape}
	
	% 中文字体
	{\songti 宋体} \quad{heiti 黑体} \quad{\fangsong 仿宋} \quad {\kaishu 楷书}	% \quad表示空格
	
	中文字体的\textbf{粗体}与\textit{斜体}
	
	% 字体大小 以normalsize为参考,可在\documentclass[]中配置
	{\tiny  		Hello }\\
	{\scriptsize  	Hello }\\
	{\footnotesize  Hello }\\
	{\small  		Hello }\\
	{\normalsize 	Hello }\\
	{\large  		Hello }\\
	{\Large  		Hello }\\
	{\LARGE  		Hello }\\
	{\huge  		Hello }\\ 
	
	% 中文字号设置命令
	\zihao{5}你好!
	\myfont	% latex格式与内容分离
	
	
\end{document}