% 文档基本结构
\documentclass{ctexart}
% \usepackage{ctex}



\title{First Tex File}
\author{Bayyy}
\date{\today}

% ======设置标题的格式======
\CTEXsetup{section = {
		format+ = \zihao{-4} \heiti \raggedright,
		name = {,、},
		number = \chinese{section},
		beforeskip = 1.0ex plus 0.2ex minus .2ex,
		afterskip = 1.0ex plus 0.2ex minus .2ex,
		aftername = \hspace{0pt}
	},
	subsection = {
		format+ = \zihao{5} \heiti \raggedright,
		% name = {\thesubsetion、}
		name = {,、},
		number = \arabic{subsection},
		beforeskip = 1.0ex plus 0.2ex minus .2ex,
		afterskip = 1.0ex plus 0.2ex minus .2ex,
		aftername = \hspace{0pt}
	}
}

% 正文区(文稿区)
\begin{document}
	\maketitle	% 使得导言区的设置生效
	
	\section{引言}
	中国人口模式的转变发生于民国时期 关于民国的进步,我只讲两个过去人们比较忽略的问题。一是人口模式。如前所述,传统时代人口的增减是王朝兴衰的显示器。
	% 换行的多种方式
	% 1. 空行(一个或多个效果类似)*
	% 2. \\	(不产生新的段落)
	% 3. \par (产生新的段落)
	\section{实验方法}
	\section{实验结果}
	\subsection{数据}
	\subsection{图表}
	\subsubsection{实验过程}
	\section{结论}
	\section{致谢}	
	
\end{document}
